\PassOptionsToPackage{table}{xcolor}
\documentclass[10pt]{beamer}
\usepackage[english]{babel}

% Beamer theme
\usetheme{metropolis}

\usepackage{smartdiagram}
\usepackage{lstautogobble}
\usepackage{listings}
\usepackage{breqn}
\usepackage{amsmath}
\usepackage{booktabs}
\usepackage[scale=2]{ccicons}%creative commons
\setbeamercovered{transparent}%invisible by default
\usepackage{array}
\newcolumntype{L}[1]{>{\raggedright\let\newline\\arraybackslash\hspace{0pt}}m{#1}}
\newcolumntype{C}[1]{>{\centering\let\newline\\arraybackslash\hspace{0pt}}m{#1}}
\newcolumntype{R}[1]{>{\raggedleft\let\newline\\arraybackslash\hspace{0pt}}m{#1}}

\usepackage{pgfplots}
\usepgfplotslibrary{dateplot}
\usepackage{tikz}
\usetikzlibrary{positioning,chains,fit,shapes,calc}
\usetikzlibrary{positioning,chains,fit,shapes,calc,automata,positioning}
\usepackage{fancyvrb}

% To include metapost files.
\usepackage{ifpdf}                        % To check if pdflatex is used.
\ifpdf
  \DeclareGraphicsRule{*}{mps}{*}{}
\fi

% Define path for images.
\graphicspath{{./}, {./Images/}}

% few useful commands
\providecommand{\ie}{i.\,e.}
\providecommand{\Ie}{I.\,e.}
\providecommand{\eg}{e.\,g.}
\providecommand{\Eg}{E.\,g.}

\providecommand{\N}{\mathbb{N}}

% Basic listing setting (eg, autogobble to remove left margin)
\lstset{basicstyle=\ttfamily, autogobble}

% metropolis template settings
\metroset{block=fill}
\metroset{titleformat frame=smallcaps}

\title{Rockin' in a \emph{free} world}
%\subtitle{A tour of free structures}

\date{\today}
\author[Alessandro Candolini]{Alessandro Candolini}
%\institute{}
% \titlegraphic{\hfill\includegraphics[height=1.5cm]{logo/logo}}

\begin{document}

\maketitle

\begin{frame}{Agenda}
  \setbeamertemplate{section in toc}[sections numbered]
  \tableofcontents[hideallsubsections]
\end{frame}


\section{Free monoids}








\begin{frame}
  \begin{definition}[monoid, set-theoretical]

    A ``monoid'' is a tuple $(A, \varphi, e)$ where
    \begin{itemize}
      \item A is a set\footnote{For all practical purposes, it's convenient to restrict the definition to non-empty sets.}
      \item $\varphi : A \times A \rightarrow A$ is a binary \emph{associative} operation on A
      \item $e \in A$ is a ``neutral'' element, ie, for every $a \in A $, $\varphi(a, e) = \varphi(e,a) = a$
    \end{itemize}

  \end{definition}

\end{frame}

\begin{frame}[fragile]
  \begin{lstlisting}[language=haskell]
  class Monoid a where
     (<>) :: a -> a -> a -- binary operation
     mempty :: a  -- neutral element
  \end{lstlisting}
\end{frame}


\begin{frame}[fragile]
\begin{lstlisting}[language=haskell]
  class Semigroup a where
    (<>) :: a -> a -> a -- binary operation

  class Semigroup a => Monoid a where
    mempty :: a -- neutral element
\end{lstlisting}
\end{frame}

\begin{frame}
  Monoids are everywhere.

  Classic examples:
  \begin{itemize}
    \item $(\N, +, 0)$ is a (commutative) monoid
    \item $(\N, \times , 1)$ is a (commutative) monoid
    \item String concatenation is a (non-commutative) monoid, with empty string as neutral element
    \item Singly-linked list concatenation is a (non-commutative) monoid, with empty list as neutral element
    \item etc
  \end{itemize}
\end{frame}


\begin{frame}[fragile]
  \begin{lstlisting}[language=haskell]
  {-# LANGUAGE DerivingVia #-}

  newtype AdditiveInteger = AdditiveInteger Integer
    deriving (Eq,Show)
    deriving Num via Integer

  instance Monoid AdditiveInteger where
     (<>) = (+)
     mempty = 0
  \end{lstlisting}
\end{frame}
\begin{frame}[fragile]
  \begin{lstlisting}[language=haskell]
  {-# LANGUAGE DerivingVia #-}

  newtype MultiplicativeInteger = MultiplicativeInteger Integer
    deriving (Eq,Show)
    deriving Num via Integer

  instance Monoid MultiplicativeInteger where
     (<>) = (*)
     mempty = 1
  \end{lstlisting}
\end{frame}

\begin{frame}[fragile]
  \begin{lstlisting}[language=haskell]
  {-# LANGUAGE FlexibleInstances #-}

  instance Monoid String where
     (<>) = (++)
     mempty = ""
  \end{lstlisting}
  \footnotetext{https://github.com/ghc-proposals/ghc-proposals/pull/279}
\end{frame}

\begin{frame}[fragile]
  ``Homeworks'':
  \begin{itemize}
    \item prove whether (\verb|Double| , $+$, $0$) is a Monoid in Scala or not
    \item prove whether (\verb|BigDecimal| , $+$, $0$) is a Monoid in Scala or not
    \item prove whether sorters on a list of sortable elements can be equipped with a Monoid instance or not.
    \item what about filter predicates?
  \end{itemize}
\end{frame}

\begin{frame}[fragile]
  \begin{lstlisting}[language=haskell]
  data Erratum = BannedHashtag 
     | PriceLowerThanShippingCost 
     | TooManyHashtags 
     | .. deriving (Eq, Show)
  \end{lstlisting}
\end{frame}

\begin{frame}[fragile]
  \begin{lstlisting}[language=haskell]
  instance Monoid [a] where
     (<>) = (++)
     mempty = []
  \end{lstlisting}
\end{frame}


\begin{frame}[fragile]
  \verb|[Erratum]| forms a monoid
\end{frame}

\begin{frame}
  (((()()()(()))(())()()((())()()()((
\end{frame}
\begin{frame}[fragile]
  \begin{dmath*}
    A = \left\{ a,b,e \right\}
  \end{dmath*}
\end{frame}

\begin{frame}[fragile]
  \begin{align*}
    \varphi(a, e) &= \varphi(e, a)  = a\\
    \varphi(b, e) &= \varphi(e, b)  = a\\
    \varphi(a,b) &= e \\
    \varphi(a,b) &\neq e 
  \end{align*}
\end{frame}


\begin{frame}[fragile]
  \begin{lstlisting}[language=haskell]
  data Bicyclic = Bicyclic Int Int 
        deriving (Eq, Show)

  instance Semigroup Bicyclic where
    (<>) (Bicyclic a b) (Bicyclic c d)
       | b <= c = Bicyclic (a + c -b ) d
       | otherwise = Bicyclic a (d + b - c)

  instance Monoid Bicyclic where
    mempty = Balance 0 0
  \end{lstlisting}
\end{frame}

\begin{frame}[fragile]
  \begin{lstlisting}[language=haskell]
  transform :: Char -> Bicyclic
  transform '(' = Bicyclic 0 1
  transform ')' = Bicyclic 1 0
  transform _ = mempty

  balance :: String -> Bicyclic
  balance = foldMap transform

  balanced = (mempty ==) . balance
  \end{lstlisting}
\end{frame}
\begin{frame}[fragile]
  \begin{lstlisting}[language=haskell]
  import Test.Hspec
  
  spec :: Spec
  spec = describe "Monoidal parsing" $ do
   it "()" $
    balance "()" `shouldBe` Balance 0 0
   it ")(" $
    balance ")(" `shouldBe` Balance 1 1
   it "))" $
    balance "))" `shouldBe` Balance 2 0
   it "((" $
    balance "((" `shouldBe` Balance 0 2S

   it "()((())()(()))(())" $
    balance "()((())()(()))(())" `shouldBe` 
       Balance 0 0
  \end{lstlisting}
\end{frame}

\section{Free monads}
\begin{frame}[fragile]
  \begin{lstlisting}[language=haskell]
  import Data.Time
  
  data Todo = Todo String (Maybe UTCTime) 
    deriving (Eq,Show) 
  \end{lstlisting}
\end{frame}

\begin{frame}[fragile]
  \begin{lstlisting}[language=haskell]
  program :: Todo -> IO ()
  program t@(Todo _ Nothing) = persist t
  program t@(Todo _ (Just due))=
    getCurrentTime >>= \ now ->
    if now >= due then
       persist t
    else
       pure () -- ignore errors  

  -- somewhere 
  persist :: Todo -> IO()
  persist = undefined
       
  \end{lstlisting}
\end{frame}

\begin{frame}[fragile]
  \begin{lstlisting}[language=haskell]
   {-# LANGUAGE DeriveFunctor #-}

    data Program a =
      Persist Todo (Program a) |
      Done a deriving (Eq,Show, Functor)
  \end{lstlisting}
\end{frame}

\begin{frame}[fragile]
  \begin{lstlisting}[language=haskell]
    program' :: Todo -> Program ()
    program' t = Persist t (pure ())
  \end{lstlisting}
\end{frame}



\begin{frame}[fragile]
  \begin{lstlisting}[language=haskell]
    data Program a =
      GetCurrentTime (UTCTime -> Program a) |
      Persist Todo (Program a) |
      Done a deriving (Functor)
  \end{lstlisting}
\end{frame}


\begin{frame}[fragile]
  \begin{lstlisting}[language=haskell]
  program' :: Todo -> Program ()
  program' t@(Todo _ Nothing) = Persist t (pure ())
  program' t@(Todo _ (Just due))=
    GetCurrentTime (\ now ->
    if now >= due then
       Persist t (pure ())
    else
       pure () -- ignore errors
       )
  \end{lstlisting}
\end{frame}

\begin{frame}[fragile]
  \begin{lstlisting}[language=haskell]
  data Free f a
    = Pure a
    | Free (f (Free f a)) deriving (Functor)


  liftF :: Functor f => f a -> Free f a
  liftF = Free . fmap Pure
    
  \end{lstlisting}
\end{frame}

\begin{frame}[fragile]
  \begin{lstlisting}[language=haskell]
  instance Functor f => Monad (Free f) where
    return = Pure
    Pure x  >>= g  =  g x
    Free fx >>= g  =  Free ((>>= g) <$> fx)
  \end{lstlisting}
\end{frame}

\begin{frame}[fragile]
  \begin{lstlisting}[language=haskell]
  data ProgramF a =
    GetCurrentTime (UTCTime -> Program a) |
    Persist Todo (Program a) |
    Done a deriving (Functor)
  \end{lstlisting}
\end{frame}

\begin{frame}[fragile]
  \begin{lstlisting}[language=haskell]
  type Program = Free ProgramF
  \end{lstlisting}
\end{frame}

\begin{frame}[fragile]
  \begin{lstlisting}[language=haskell]
  \end{lstlisting}
\end{frame}

\begin{frame}[fragile]
  Advantages:
  \begin{itemize}
    \item Managing capabilities precisely
    \item Better separation of business logic and effectful interpreters
  \end{itemize}
\end{frame}


\begin{frame}[fragile]
  \begin{lstlisting}[language=haskell]
  interface Clock {
     def now : Instant 
  }   
  \end{lstlisting}
\end{frame}

\begin{frame}[fragile]
  Boilerplate:
  \begin{itemize}
    \item lifting of operations (can be absorbed with metaprogramming) 
    \item Multiple algebras / capabilities
    \item Copy paste code in almost similar interpreters 
    \item test interpreters (pure ones, eg, State) 
  \end{itemize}
\end{frame}

\section{Polysemy}
\section{Free applicatives and optparse}

\plain{Questions?}

%\begin{frame}[allowframebreaks] {References}
% \bibliography{demo}
% \bibliographystyle{abbrv}
%\end{frame}

\end{document}
%% Useful for copy pasting. Would be better to just have snippets
\begin{frame}[fragile]
  \frametitle{}
  \begin{lstlisting}[language=scala, basicstyle=\ttfamily]
  \end{lstlisting}
\end{frame}
