\PassOptionsToPackage{table}{xcolor}
\documentclass[10pt]{beamer}
\usepackage[english]{babel}

% Beamer theme
\usetheme{metropolis}

\usepackage{smartdiagram}
\usepackage{lstautogobble}
\usepackage{listings}
\usepackage{booktabs}
\usepackage[scale=2]{ccicons}%creative commons
\setbeamercovered{transparent}%invisible by default
\usepackage{array}
\newcolumntype{L}[1]{>{\raggedright\let\newline\\arraybackslash\hspace{0pt}}m{#1}}
\newcolumntype{C}[1]{>{\centering\let\newline\\arraybackslash\hspace{0pt}}m{#1}}
\newcolumntype{R}[1]{>{\raggedleft\let\newline\\arraybackslash\hspace{0pt}}m{#1}}

\usepackage{pgfplots}
\usepgfplotslibrary{dateplot}
\usepackage{tikz}
\usetikzlibrary{positioning,chains,fit,shapes,calc}
\usetikzlibrary{positioning,chains,fit,shapes,calc,automata,positioning}
\usepackage{fancyvrb}

% To include metapost files.
\usepackage{ifpdf}                        % To check if pdflatex is used.
\ifpdf
  \DeclareGraphicsRule{*}{mps}{*}{}
\fi

% Define path for images.
\graphicspath{{./}, {./Images/}}

% few useful commands
\providecommand{\ie}{i.\,e.}
\providecommand{\Ie}{I.\,e.}
\providecommand{\eg}{e.\,g.}
\providecommand{\Eg}{E.\,g.}

\providecommand{\N}{\mathbb{N}}

% Basic listing setting (eg, autogobble to remove left margin)
\lstset{basicstyle=\ttfamily, autogobble}

% metropolis template settings
\metroset{block=fill}
\metroset{titleformat frame=smallcaps}

\title{Rockin' in a \emph{free} world}
%\subtitle{}

\date{\today}
\author[Alessandro Candolini]{Alessandro Candolini}
%\institute{}
% \titlegraphic{\hfill\includegraphics[height=1.5cm]{logo/logo}}

\begin{document}

\maketitle

\begin{frame}{Agenda}
  \setbeamertemplate{section in toc}[sections numbered]
  \tableofcontents[hideallsubsections]
\end{frame}


\section{Free monoids}








\begin{frame}
  \begin{definition}[monoid]

    A ``monoid'' is a tuple $(A, \varphi, e)$ where
    \begin{itemize}
      \item A is a set\footnote{For all practical purposes, it's convenient to restrict the definition to non-empty sets.}
      \item $\varphi : A \times A \rightarrow A$ is an \emph{associative} binary operation on A
      \item $e \in A$ is a ``neutral'' element, ie, for every $a \in A $, $\varphi(a, e) = \varphi(e,a) = a$
    \end{itemize}

  \end{definition}

\end{frame}

\begin{frame}[fragile]
  \begin{lstlisting}[language=haskell]
  class Monoid a where
     (<>) :: a -> a -> a -- binary operation
     mempty :: a  -- neutral element
  \end{lstlisting}
\end{frame}


\begin{frame}[fragile]
\begin{lstlisting}[language=haskell]
  class Semigroup a where
    (<>) :: a -> a -> a -- binary operation

  class Semigroup a => Monoid a where
    mempty :: a -- neutral element
\end{lstlisting}
\end{frame}

\begin{frame}
  Monoids are everywhere:
  \begin{itemize}
    \item $(\N, +, 0)$ is a (commutative) monoid
    \item $(\N, \times , 1)$ is a (commutative) monoid
    \item String concatenation is a (non-commutative) monoid, with empty string as neutral element
    \item Singly-linked list concatenation is a (non-commutative) monoid, with empty list as neutral element
    \item etc
  \end{itemize}
\end{frame}

\begin{frame}[fragile]
  \begin{lstlisting}[language=haskell]
  {-# LANGUAGE DerivingVia #-}

  newtype AdditiveInteger = AdditiveInteger Integer
    deriving (Eq,Show)
    deriving Num via Integer

  instance Monoid AdditiveInteger where
     (<>) = (+)
     mempty = 0
  \end{lstlisting}
\end{frame}
\begin{frame}[fragile]
  \begin{lstlisting}[language=haskell]
  {-# LANGUAGE DerivingVia #-}

  newtype MultiplicativeInteger = MultiplicativeInteger Integer
    deriving (Eq,Show)
    deriving Num via Integer

  instance Monoid MultiplicativeInteger where
     (<>) = (*)
     mempty = 1
  \end{lstlisting}
\end{frame}

\begin{frame}[fragile]
  \begin{lstlisting}[language=haskell]
  {-# LANGUAGE FlexibleInstances #-} 
  -- https://github.com/ghc-proposals/ghc-proposals/pull/279

  instance Monoid String where
     (<>) = (++)
     mempty = ""
  \end{lstlisting}
\end{frame}

\begin{frame}[fragile]
  \begin{lstlisting}[language=haskell]
  instance Monoid [a] where
     (<>) = (++)
     mempty = []
  \end{lstlisting}
\end{frame}

\begin{frame}
  Monoids, categorically


\end{frame}


\section{Free monads}
\begin{frame}[fragile]
  \frametitle{Placeholder Example}
\end{frame}




\section{Polysemy}


\section{Free monoidals}


\section{Optparse}

\plain{Questions?}

%\begin{frame}[allowframebreaks] {References}
% \bibliography{demo}
% \bibliographystyle{abbrv}
%\end{frame}

\end{document}
%% Useful for copy pasting. Would be better to just have snippets
\begin{frame}[fragile]
  \frametitle{}
  \begin{lstlisting}[language=scala, basicstyle=\ttfamily]
  \end{lstlisting}
\end{frame}
